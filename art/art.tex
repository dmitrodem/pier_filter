\documentclass{pj}
\usepackage{graphics}

\begin{document}

\setcounter{page}{1}
\pjheader{Vol.\ x, y--z, 2014}

\title
{Replace This Line with the Title of Your Manuscript}
\footnote{\it Received date}
\footnote{\hskip-0.12in*\, Corresponding author:~Dmitriy~Dyomin (demin.da@mipt.ru).}
\footnote{\hskip-0.12in\textsuperscript{1} The first affiliation. \textsuperscript{2} The second affiliation.}

\author{Dmitriy~A.~Dyomin\textsuperscript{*, 1} and FirstName2~LastName2\textsuperscript{2}}



\begin{abstract}
  Abstract
\end{abstract}
\section{Introduction}
\label{sec:introduction}

One of the most popular ways to synthesize filter is to specify its
return loss. For the case of linear time-invariant networks (such as
``ladder'' networks consisting of inductive and capacitive lumped
elements), the transfer function is known to be rational function,
i.e.

\[
  S_{21}(p) = \frac{N(p)}{M(p)}
\]

where $N(p)$ and $M(p)$ are polynomial and $p = \sigma + j \omega$ is
a complex frequency variable.

A classic approach of synthesizing lowpass filter is done by specifing
explicit frequency dependence of return loss
$\left|S_{21}(j\omega)\right|^2$. Using that information, one can
determine polynomials $N(p)$ and $M(p)$ and (by finding their's roots)
zeros and poles of the transfer function.

Unfortunately, there is no easy way to calculate filter prototype
values $g_i$ using its transfer function. Explicit solutions are
well-known for a case of maximally flat response filter
(a.k.a. Butterworth filter) and equiripple filter (or Tchebysheff
filter).



In a classic case of a ``ladder'' network made of
inductive and capacitive lumped elements, insertion loss may be
represented as $P_{LR} = 1 + M(\omega^2)/N(\omega^2)$. Filter design
is accomplished by specifying explicit polynoms $M(\omega^2)$,
$N(\omega^2)$. Filter design tools such as {\em Matlab Filter Design
  Toolbox} or {\em scipy.signals} provide means to synthesize filter
transfer function $H(s)$, namely its zeros and poles.



\end{document}
%%% Local Variables:
%%% mode: latex
%%% TeX-master: t
%%% End:
